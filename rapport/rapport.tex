% !TEX TS-program = pdflatex
% !TEX encoding = UTF-8 Unicode

% This is a simple template for a LaTeX document using the "article" class.
% See "book", "report", "letter" for other types of document.

\documentclass[11pt]{article} % use larger type; default would be 10pt

\usepackage[utf8]{inputenc} % set input encoding (not needed with XeLaTeX)

%%% Examples of Article customizations
% These packages are optional, depending whether you want the features they provide.
% See the LaTeX Companion or other references for full information.

%%% PAGE DIMENSIONS
\usepackage{geometry} % to change the page dimensions
\geometry{a4paper} % or letterpaper (US) or a5paper or....
\geometry{margin=1in} % for example, change the margins to 2 inches all round
% \geometry{landscape} % set up the page for landscape
%   read geometry.pdf for detailed page layout information

\usepackage{import}
\subimport{../common/}{header}

% \usepackage[parfill]{parskip} % Activate to begin paragraphs with an empty line rather than an indent

%%% PACKAGES
\usepackage{booktabs} % for much better looking tables
\usepackage{array} % for better arrays (eg matrices) in maths
\usepackage{paralist} % very flexible & customisable lists (eg. enumerate/itemize, etc.)
\usepackage{verbatim} % adds environment for commenting out blocks of text & for better verbatim
\usepackage{subfig} % make it possible to include more than one captioned figure/table in a single float
% These packages are all incorporated in the memoir class to one degree or another...

%%% HEADERS & FOOTERS
\usepackage{fancyhdr} % This should be set AFTER setting up the page geometry
\pagestyle{fancy} % options: empty , plain , fancy
\renewcommand{\headrulewidth}{0pt} % customise the layout...
\lhead{}\chead{}\rhead{}
\lfoot{}\cfoot{\thepage}\rfoot{}

%%% SECTION TITLE APPEARANCE
\usepackage{sectsty}
\allsectionsfont{\sffamily\mdseries\upshape} % (See the fntguide.pdf for font help)
% (This matches ConTeXt defaults)

%%% ToC (table of contents) APPEARANCE
\usepackage[nottoc,notlof,notlot]{tocbibind} % Put the bibliography in the ToC
\usepackage[titles,subfigure]{tocloft} % Alter the style of the Table of Contents
\renewcommand{\cftsecfont}{\rmfamily\mdseries\upshape}
\renewcommand{\cftsecpagefont}{\rmfamily\mdseries\upshape} % No bold!

%%% END Article customizations

%%% The "real" document content comes below...

\title{Distance maximale de visibilité}
\author{}
\date{} % Activate to display a given date or no date (if empty),
         % otherwise the current date is printed 

\begin{document}
\maketitle

\section{Calculs}

\subsection{Seuil scopique de vision}

Le seuil scopique absolu de visibilité est donné par  :
\begin{equation}
I \geq I_{min} \sim 2,5 \times 10^{-6} \mbox{  lumen/m}^2
\end{equation}

Où $I_{min}$ est le flux lumineux minimum pouvant être perçu pour une source fixe éclairant constamment.

Par ailleurs le flux lumineux en lumen/m$^2$ (ou "lux") est donné par :
\begin{equation}
I = \dfrac{L}{d^2} = \dfrac{L_0}{d^2}  \dint_{0}^{+\infty} V(\lambda) p(\lambda) \dd \lambda
\label{eq:integral}
\end{equation}

Où $L_0 = 1700 \mbox{lumen/W}$ en vision scopique, $V$ est l'efficacité lumineuse spectrale relative, et $p(\lambda)$ la puissance du signal lumineux (en W) comprise dans l'invervalle $[\lambda, \lambda+\dd\lambda[$.

La fonction $V$ a l'allure donnée sur la figure (\ref{fig:efficacite_spectrale}). On réalise un fit gaussien des données expérimentales afin d'en obtenir une interpolation fiable qui permette le calcul de l'intégrale \eqref{eq:integral}.

\begin{figure}[H]
\centering
  \caption{Efficacité lumineuse spectrale et fit $V(\lambda)$ sous la forme $\exp \left ( (\lambda-\lambda_0)^2/2\Delta\lambda^2\right )$. On trouve $\lambda_0 = $ 503 nm, $\Delta \lambda^2 =$ 40 nm.
\label{fig:efficacite_spectrale}}
\input{figures/efficacite}
\end{figure}

\subsection{Corps noir}

Les étoiles sont des corps noirs. Aussi il est possible de considérer, si on néglige les phénomènes atmosphériques, que $p(\lambda)$ est simplement le spectre d'un corps noir (loi de planck). Dans ce cas, la valeur de l'intégrale ne dépend que de la température et du rayon de l'étoile :

\begin{equation}
I =  4\pi \dfrac{R_{*}}{d^2} ^2 I_{*}(T) = 4\pi L_0 \dfrac{R_{*}^2}{d^2} \sigma T_{*}^4 \eta(T)
\label{eq:integral_blackbody}
\end{equation}

Ici $\eta$ est une fonction sans dimension qui représente le "rendement lumineux" de l'étoile :

Et par ailleurs :
\begin{equation}\eta (T) = \dfrac{1}{\sigma T_{*}^4}  \dint_{0}^{+\infty} \dfrac{2\pi h c^2}{\lambda^5} \dfrac{e^{\left( (\lambda-\lambda_0)^2/2\Delta \lambda^2\right )}}{e^{hc/(\lambda kT)}-1} H(\lambda) \dd \lambda
\end{equation}

La fonction $\lambda \mapsto H(\lambda)$ représente la fraction de rayonnement transmise à travers l'atmosphère (à la verticale par exemple).

$\eta$ est nécessairement plus petit que 1. Dans le cas idéal d'un rayonnement monochromatique à la longueur d'onde de sensibilité maximum de l'oeil $\eta = 1$. Pour un corps noir cependant le spectre est clairement non monochromatique et le rendement plus faible que 1. La courbe de $\eta(T)$ est donnée figure $\ref{fig:eta}$.

\subsection{Atmosphère}

La courbe de transmission de l'atmosphère à la verticale est donnée par la figure suivante :

\begin{figure}[H]
\centering
  \caption{Courbe de transmission $H(\lambda)$ de l'atmosphère (traversée à la verticale)
\label{fig:atmo}}
\input{figures/atmosphere.tex}
\end{figure}

$H$ peut s'obtenir si l'on connait la quantité de rayonnement absorbée par unité d'épaisseur d'atmosphère en fonction de l'altitude $\alpha(x, \lambda)$ :
\begin{equation}
H(\lambda) = K \exp \left [ -\dint_0^{+\infty} \alpha(x, \lambda) \dd l\right ]
\end{equation}

Si l'atmosphère est traversée "de biais" : 

\begin{equation}
H_\theta(\lambda) = K \exp \left [ -\dint_0^{+\infty} \alpha(x(l), \lambda) \dd l\right ] \mbox{ et } x = \sqrt{R^2 + l^2 + 2 R l \sin \theta} - R
\end{equation}

De là : 
\begin{equation}
l^2 + 2 R l \sin \theta + R^2 - (x+R)^2 = 0
\end{equation}

\begin{equation}
\Delta = 4 R^2 \sin^2 \theta - 4 R^2 + 4 (x+R)^2 = 4 \left [ (x+R)^2 -  R^2\cos^2 \theta \right ]
\end{equation}

\begin{equation}
l = - R \sin \theta + \sqrt{(x+R)^2 -  R^2\cos^2 \theta }
\end{equation}

\begin{equation}
\dd l =  \dfrac{x + R}{\sqrt{(x+R)^2 -  R^2\cos^2 \theta }} \dd x
\end{equation}

\begin{equation}
H_\theta(\lambda) = K \exp \left [ -\dint_0^{+\infty} \alpha(x, \lambda) \dfrac{x + R}{\sqrt{(x+R)^2 -  R^2\cos^2 \theta }} \dd x\right ] 
\end{equation}

En supposant que l'atmosphère est de dimension $h$ très inférieure au rayon terrestre, et à l'horizon ($\theta = 0$) :

\begin{equation}
H_{\mbox{hor}}(\lambda) \simeq K \exp \left [ -\dint_0^{h} \alpha(x, \lambda) \sqrt{\dfrac{R}{2x}} \dd x\right ] 
\end{equation}

On suppose par ailleurs que l'atmosphère possède un profil isotherme et que $\alpha(x, \lambda) = \alpha_0(\lambda) \exp(-x/L)$ ($L \ll h)$. Dès lors :

\begin{equation}
H_{\mbox{hor}}(\lambda) \simeq K \exp \left [ - \sqrt{\dfrac{R}{2}} \alpha_0 (\lambda) \dint_0^{+\infty} \dfrac{\exp(-x/L)}{\sqrt{x}} \dd x\right ] = K \exp \left [ - \sqrt{\dfrac{\pi R L}{2}} \alpha_0 (\lambda)\right ]
\end{equation}

Si bien que :

\begin{equation}
H_{\mbox{hor}}(\lambda) = H(\lambda)^{\sqrt{\pi R/2L}}
\end{equation}

Or $\sqrt{\pi R/2L} \sim $ 35 ! On ne voit rien à l'horizon. Il faut s'en approcher légèrement seulement.

Plus généralement, il faut intégrer numériquement :
\begin{equation}
H_\theta(\lambda) = \exp \left [ -\alpha_0(\lambda) \dint_0^{+\infty}\exp(-x/L) \dfrac{x + R}{\sqrt{(x+R)^2 -  R^2\cos^2 \theta }} \dd x\right ] 
\end{equation}

Et alors :

\begin{equation}
\mbox{AM}(\theta) \equiv \dfrac{\ln H_\theta(\lambda)}{\ln H(\lambda)} = \dfrac{1}{L}\dint_0^{+\infty}\exp(-x/L) \dfrac{x + R}{\sqrt{(x+R)^2 -  R^2\cos^2 \theta }} \dd x
\end{equation}

%\begin{equation}
%\dfrac{\ln H_\theta(\lambda)}{\ln H_\phi(\lambda)} =\dfrac{\ln H_\theta(\lambda)}{\ln H(\lambda)} \dfrac{\ln H(\lambda)}{\ln H_\phi(\lambda)} 
%\end{equation}

\subsection{Rendement lumineux corps noir}

\begin{figure}[H]
\centering
  \caption{rendement lumineux $\eta$ d'un corps noir en fonction de sa température. La luminosité apparente pour l'oeil d'un corps noir est donné par le produit de sa puissance et de $\eta$. 
\label{fig:eta}}
\input{figures/eta}
\end{figure}

Le flux lumineux par unité de surface de l'étoile $I_*(T) = L_0 \eta(T) \sigma T^4$ (qui ne dépend donc également que de la température) en lumen/m$^2$ est représenté figure \ref{fig:eta_t4}. 

\begin{figure}[H]
\centering
  \caption{Courbe de $I_*(T) = L_0 \eta(T) \sigma T^4$, qui est directement proportionnel au flux reçu par l'oeil.
\label{fig:eta_t4}}
\input{figures/eta_t4}
\end{figure}

\subsection{Condition de visibilité}

Finalement la condition de visibilité se réécrit :

\begin{equation}
 d \leq R_{*} \sqrt{4\pi\dfrac{I_*(T)}{I_{min}}}
\end{equation}

Cette équation peut être réécrite en terme de $\eta(T)$ :
\begin{equation}
 d \leq R_{*} \sqrt{4\pi\dfrac{L_0 \sigma T^4 \eta(T)}{I_{min}}}
\end{equation}

La luminosité donne donc une limite de la forme $d \leq \theta (T) R_*$. La fonction $\theta(T)$ est représentée figure \ref{fig:theta} ainsi que les valeurs de $d/R$ pour certaines étoiles. 

\begin{figure}[H]
\centering
  \caption{Courbe de $\theta(T) =  d_{max}/R_*$. La zone verte sous la courbe correspond aux étoiles visibles.  
\label{fig:theta}}
\input{figures/theta.tex}
\end{figure}


\begin{figure}[H]
\centering
  \caption{Courbe de $\theta(T) =  d_{max}/R_* $. La zone verte sous la courbe correspond aux étoiles visibles.  
\label{fig:theta}}
\input{figures/theta_high.tex}
\end{figure}

\begin{figure}[H]
\centering
  \caption{Courbe de $\theta(T) =  d_{max}/R_* $ en tenant compte de l'atmosphère, et pour différents angles d'élévation. (de 15 à 90 deg de 5 en 5)
\label{fig:theta_atmosphere}}
\input{figures/theta_atmosphere.tex}
\end{figure}

\begin{figure}[H]
\centering
  \caption{Courbe de $\theta(T) =  d_{max}/R_* $ en tenant compte de l'atmosphère, et pour différents angles d'élévation. (de 15 à 90 deg de 5 en 5)
\label{fig:theta_high_atmosphere}}
\input{figures/theta_high_atmosphere.tex}
\end{figure}

\subsection{Magnitude apparente}

La magnitude apparente dans une bande donnée est définie par :

\begin{equation}
m \equiv m_0 - 2.5 \log \dfrac{F}{F_0}
\end{equation}

Où $m_0$ est la magnitude d'un astre de référence dans cette bande, et $F$ et $F_0$ les flux reçus par l'astre et la référence dans cette bande.

Une bande peut être définie par une gaussienne $(\lambda_{\mbox{max}},\sigma_\lambda)$. On travaillera avec les bandes V ($(\lambda_{\mbox{max}},\sigma_\lambda) = $ (551 nm, 88 nm)) et B ($(\lambda_{\mbox{max}},\sigma_\lambda) = $ (445 nm, 94 nm))

Pour des corps noirs, en définissant le rendement dans une bande $X$ donnée comme $\eta_X(T)$ alors :

\begin{equation}
m \equiv m_0 - 2.5 \log \dfrac{\eta_X(T) \sigma T^4}{\eta_X(T_0)\sigma T_0^4} \dfrac{R^2}{R_0^2} \dfrac{d_0^2}{d^2}
\end{equation}

Cela donne en particulier :

\begin{equation}
m_{\mbox{max}} \equiv m_0 - 2.5 \log \left ( \dfrac{\eta_X(T)}{\eta_X(T_0)} \dfrac{T^4}{T_0^4}\right) - 5 \log {\dfrac{d_0}{R_0 \theta(T)}}
\end{equation}

%
%\begin{figure}[H]
%\centering
%  \caption{$\mu_b(z)$ SN Ia et best fit ($\Omega_m = $ 0,25 et $\Omega_v = $ 0,75)}
%\input{figures/snIa_bestfit}
%\end{figure}
\end{document}